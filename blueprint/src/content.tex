% In this file you should put the actual content of the blueprint.
% It will be used both by the web and the print version.
% It should *not* include the \begin{document}
%
% If you want to split the blueprint content into several files then
% the current file can be a simple sequence of \input. Otherwise It
% can start with a \section or \chapter for instance.

\chapter{definitions}

\begin{definition}
  \label{def:NCPS}
  A \textit{non-commutative probability space} \((\mathcal{A},\varphi)\) consists of
  \begin{itemize}
    \item a unital (associative) algebra \(\mathcal{A}\) (over \(\mathbb{C}\)) and
    \item a linear map \(\varphi:\mathcal{A} \rightarrow \mathbb{C}\) that preserves \(1\).
  \end{itemize}
\end{definition}

\begin{definition}
  \label{def:centered}
  An element \(a\) of a probability space \(\mathbb{A}\) is called \textit{centered} when
  \(\varphi(a)=0\).
\end{definition}

in this blueprint, when we write \([n]\), we will refer to the set/type \(\{0,\ldots,n\}\)

\begin{definition}
  \label{def:alt_label}
  A finite sequence of elements \((a_j)_{j\in [n]}\) has an \textit{alternating \(I\)-labeling} when
  there exists a function \(f:[n]\rightarrow I\) such that for adjacent \(j,j' \in [n]\),
  \(f(j)\neq f(j')\) holds.
\end{definition}

\begin{definition}
  \label{def:FreeIndep}
  \uses{def:centered,def:alt_label}
  An indexed family of subalgebras \((\mathcal{A}_i)_{i\in I}\) is called \textit{free} or
  \textit{freely independent} when the following holds:
  \begin{itemize}
    \item For any finite sequence of \(n\) elements \((a_j)_{j\in [n]}\)
    \item such that these elements are all centered,
    \item given an alternating \(I\)-labeling for \((a_j)_{j\in [n]}\)
    \item such that \(a_j \in \mathcal{A}_{f(j)}\) for all \(j\in [n]\),
    \item one can conclude that the (non-commutative) product
      \(\prod_{j\in[n]}^{\rightarrow} a_j\) is balanced
  \end{itemize}
\end{definition}
